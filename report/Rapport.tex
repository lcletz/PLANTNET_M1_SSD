\documentclass[a4paper,12pt]{article}

% Définition des packages et de leurs paramètres

\usepackage{graphicx}  % Pour insérer des images
\usepackage{textpos}   % Pour positionner précisément les images
\usepackage[french]{babel}  % Utiliser le package babel pour le français
\usepackage{csquotes}         % Ajouter csquotes pour babel
\usepackage[T1]{fontenc}      % Corriger l'encodage pour le français
\usepackage[colorlinks=true, linkcolor=blue]{hyperref}  % Liens cliquables en bleu sans les encadrer
\usepackage{amssymb}    % pour les symboles mathématiques
\usepackage{float}      % Pour le placement précis des figures avec [H]
\usepackage{microtype}

\usepackage{geometry}  % Pour ajuster les marges
\geometry{top=2cm, bottom=2cm, left=2cm, right=2cm}

\usepackage[style=apa, backend=biber]{biblatex}
\addbibresource{references.bib}

\usepackage{tocloft} % Pour modifier l'apparence du sommaire
\setlength{\cftbeforesecskip}{10pt} % Espacement entre les sections
\setlength{\cftbeforesubsecskip}{8pt} % Espacement entre les sous-sections
\renewcommand{\cftsecleader}{\cftdotfill{\cftdotsep}}
\setlength{\cftaftertoctitleskip}{25pt} % Espace sous le titre du sommaire

% Informations du document

\title{Projet Pl@ntnet}
\author{Émilie Aigoin}
\date{January 2025}

\begin{document}

% Logo de l'université en haut à gauche
\begin{textblock*}{3cm}(0.01cm,0.2cm)
    \includegraphics[width=4cm]{images/Universite.png}
\end{textblock*}

% Logo du master en haut à droite
\begin{textblock*}{3cm}(11cm,1cm)
    \includegraphics[width=6cm]{images/SSD.png} 
\end{textblock*}

% Titre principal
\vspace{8cm}
\begin{center}
\large{Projet de master présenté dans le contexte de l'UE HAX817X} \\ \vspace{0.4cm}
    {\LARGE \textbf{Prédiction conformelle et base de données \\ \vspace{0.4cm} Pl@ntNet-CrowdSWE}}\\[1cm]
\end{center}

% Logo de l'application centré
\begin{textblock*}{3cm}(5cm,0.5cm)
    \includegraphics[width=6cm]{images/plantenet.png}  % Remplacer par le bon fichier
\end{textblock*}

% Informations
\vfill
\begin{center}
Présenté par \\ \vspace{0.2cm}
    {\textbf{AIGOIN Emilie \\ \vspace{0.1cm} CLETZ Laura \\ \vspace{0.1cm} THOMAS Anne-Laure}}\\ \vspace{0.6cm}
    
Sous la direction (ou co-direction) de \\ \vspace{0.2cm}
    {\textbf{BOTELLA Christophe \\ \vspace{0.1cm} SALMON Joseph }}\\ \vspace{1.5cm}
    
    {\large Master Statistique et Science des Données, \\ \vspace{0.1cm} Université de Montpellier}\\ \vspace{0.6cm}
    {\large Année 2024 - 2025}
\end{center}

% Page de garde sans numéro de page
\thispagestyle{empty}

\newpage

% Insertion du sommaire
\tableofcontents 

\newpage

%%%%%%%%%%%%%%%%%%%%%%%%%%%%%%%%%%%%%%%%%%%%%%%%%%%%%%%%%%%%%%%%%%%%%%%%%%%%%%%%
%%%%%%%%%%%%%%%%%%%%%%%%%%%%%%%%%%%%%%%%%%%%%%%%%%%%%%%%%%%%%%%%%%%%%%%%%%%%%%%%

\section{Introduction}

La reconnaissance des plantes est une problématique clé en botanique, avec des applications directes dans la conservation de la biodiversité, la gestion des écosystèmes ou encore la culture personnelle. Parmi les initiatives les plus importantes dans ce domaine, le projet Pl@ntNet joue un rôle central en proposant une application de reconnaissance automatique des plantes basée sur des données collectées par les utilisateurs. Cette approche de science participative permet de constituer une base de données massive et diversifiée, essentielle pour entraîner et affiner les modèles de classification.

\vspace{0.2cm}

Cependant, garantir la fiabilité des prédictions reste un défi. Les données collectées, bien que nombreuses, peuvent être de qualité variable, et la diversité des espèces ainsi que les conditions de prise de vue (lumière, angle, netteté) introduisent des incertitudes. Pour améliorer la précision de l'application, il ne suffit pas d’optimiser la classification : nous devons également quantifier l’incertitude des prédictions et adapter la sortie du modèle en conséquence.

\vspace{0.2cm}

Dans ce travail, nous nous sommes intégrées au projet Pl@ntNet, avons récupéré et analysé les données issues de l’application, et avons terminé en cherchant à améliorer la fiabilité des prédictions en les rendant adaptatives. Plutôt que de fournir une unique réponse avec une probabilité associée, notre objectif était de générer un ensemble de prédictions dont la probabilité de contenir la bonne espèce était presque garantie. Cet ensemble devait être le plus restreint possible pour éviter les suggestions inutiles, mais devait aussi s’ajuster en fonction de la difficulté de l’identification : être plus précis pour les cas évidents et plus large pour les situations ambiguës.

\vspace{0.2cm}

Pour répondre à cette problématique, nous allons utiliser la prédiction conforme, une approche statistique permettant de garantir un niveau de confiance contrôlé sur les résultats du modèle. Dans la suite de ce rapport, nous détaillerons comment cette méthode peut être appliquée aux données de Pl@ntNet pour améliorer la robustesse et la pertinence des prédictions.

%%%%%%%%%%%%%%%%%%%%%%%%%%%%%%%%%%%%%%%%%%%%%%%%%%%%%%%%%%%%%%%%%%%%%%%%%%%%%%%%
%%%%%%%%%%%%%%%%%%%%%%%%%%%%%%%%%%%%%%%%%%%%%%%%%%%%%%%%%%%%%%%%%%%%%%%%%%%%%%%%

\section{Pl@ntnet}

%%%%%%%%%%%%%%%%%%%%%%%%%%%%%%%%%%%%%%%%%%%%%%%%%%%%%%%%%%%%%%%%%%%%%%%%%%%%%%%%

\subsection{Application Pl@ntnet}

Pl@ntNet est un projet de sciences participatives accessible sous forme d’application mais également avec une \href{https://identify.plantnet.org/fr}{interface en ligne}. Son objectif est d'aider à identifier une espèce de plante à partir d'une photographie. Elle se base sur l’intelligence artificielle pour faciliter l’identification et l’inventaire des espèces végétales qu'elle recense géographiquement en fonction de la localisation de la photo.

\vspace{0.2cm}

Pl@ntNet est basée sur un principe d’apprentissage coopératif. Les utilisateurs peuvent partager leurs photographies (que nous appellerons des observations) et celles-ci peuvent être révisées par la communauté et utilisées par l’IA pour apprendre à reconnaître les plantes. Il est par exemple possible de confirmer le nom d’une espèce ou bien d'en suggérer une autre si l’on reconnaît la plante présente sur une photographie.

\vspace{0.2cm}

Ainsi, lorsqu'un utilisateur prend une plante en photo, le logiciel génère une liste avec les noms d'espèces de plantes (appelées étiquettes) susceptibles de correspondre à celle présente sur l'image, chacune associée à une probabilité. L'espèce affichée en premier est celle que l'intelligence artificielle estime la plus probable, suivie par les suivantes classées par ordre décroissant de probabilité. 

\vspace{0.2cm}

Étant donné que nous cherchons à avoir un nombre de prédictions fini mais également raisonnable, l'affichage des étiquettes s'arrête dès que la probabilité devient inférieure à un seuil jugé trop faible pour être pertinent (ici fixé à $0,001$). De plus, nous verrons par la suite que nous cherchons également à ce que ce nombre soit adaptatif à la dureté du problème : plus il est facile de prédire l'espèce sur la photographie et moins il y aura d'étiquette en sortie.

\vspace{0.2cm}

Toutes les étiquettes que l'intelligence artificielle sort pour une observation sont stockées dans une base de données qui a constitué notre jeu de données pendant la totalité de ce travail.

%%%%%%%%%%%%%%%%%%%%%%%%%%%%%%%%%%%%%%%%%%%%%%%%%%%%%%%%%%%%%%%%%%%%%%%%%%%%%%%%

\subsection{Présentation du jeu de données}

Nous avons travaillé avec plusieurs jeux de données de différentes tailles et provenant d'observations de plantes de lieux différents.

\vspace{0.2cm}

Dans un premier temps, nous nous sommes familiarisées avec un jeu de $6 699 593$ observations d'espèces de plantes situées en Europe du Sud-Ouest (Pl@ntnet-CrowdSWE). Ces observations ont été réalisées par $823 251$ utilisateurs de l'application Pl@ntNet. Ces utilisateurs sont différenciés en deux catégories : les experts (représentent seulement $98$ utilisateurs) et les non-experts. Les experts ont été identifiés comme tels car ce sont des experts en botanique. Ainsi, quand un expert a donné son avis sur une étiquette de plante, nous admettons sa véracité. 

\vspace{0.2cm}

Ces données contiennent les variables suivantes :
\begin{itemize}
    \item Les identifiants d'utilisateur.
    \item Les identifiants d'espèce dans la base globale Pl@ntNet.
    \item Les identifiants d'espèce dans la base Pl@ntNet-CrowdSWE.
    \item Les noms scientifiques des plantes.
    \item Les espèces prédites par l'intelligence artificielle.
    \item Les probabilités associées aux espèces prédites par l'intelligence artificielle.
\end{itemize}

\vspace{0.2cm}

Elles ont été réparties dans $7$ fichiers en format JSON (JavaScript Object Notation) et $2$ fichiers textes disponibles sur la \href{https://zenodo.org/records/10782465}{plateforme Zenodo}, un centre de données du CERN ouvert à tous. Parmi les espèces enregistrées dans la base de données, $17 247$ apparaissent dans les observations et nous ont permit de réaliser une série de statistiques descriptives après croisements par identifiants des différents fichiers. 

\vspace{0.2cm}

Dans un second temps, nous avons travaillé avec un échantillon de $67 466$ résultats récoltés par l'application.

\vspace{0.2cm}

Pour chacun des numéros d'observation dans la base globale Pl@ntnet, les variables suivantes sont associées : 
\begin{itemize}
    \item Les noms scientifiques des plantes prédites.
    \item L'identifiant dans la base globale Pl@ntnet correspondant à l'espèce prédite.
    \item L'identifiant dans la base Pl@ntnet-CrowdSWE correspondant à l'espèce prédite.
    \item La probabilité que l'espèce prédite soit correcte.
\end{itemize}

Afin de pouvoir organiser et analyser ces données, nous nous sommes servis de plusieurs outils et logiciels.

%%%%%%%%%%%%%%%%%%%%%%%%%%%%%%%%%%%%%%%%%%%%%%%%%%%%%%%%%%%%%%%%%%%%%%%%%%%%%%%%

\subsection{Outils}

Afin de réaliser nos analyses, nous avons principalement utilisés les logiciels de programmation R et RStudio(\cite{RStudio}) ainsi que Python. Tous les codes que nous avons rédigés sont disponibles en open access sur \href{https://github.com/lcletz/PLANTNET_M1_SSD}{Github}.

\vspace{0.2cm}

Sous R, les packages dont nous avons fait appel sont :
\begin{itemize}
    \item \textit{jsonlite} pour l'ouverture et l'écriture de fichiers JSON.
    \item \textit{tibble} et \textit{data.table} pour une manipulation plus rapide des données.
    \item \textit{ggplot2}, \textit{gridExtra}, \textit{htmlwidget} et \textit{plotly} pour la création de rendus graphiques attrayants et interactifs.
    \item \textit{dplyr} et \textit{tidyr} pour le croisement des données.
    \item \textit{purrr} pour appliquer des fonctions prenant en entrée des vecteurs à chaque observation dans les listes larges.
\end{itemize}

\vspace{0.2cm}

Sous Python, les packages dont nous avons fait usage sont :
\begin{itemize}
    \item \textit{requests}, \textit{io}  et \textit{zipfile} pour la récupération et manipulation de fichiers ZIP.
    \item \textit{json} et \textit{tarfile} pour la manipulation de fichiers JSON et TAR (archives compressées).
    \item \textit{os} pour gérer les chemins et les fichiers.
    \item \textit{pandas} pour manipuler les données sous forme de tableaux (convertir les données JSON en data frames et les fusionner par exemple).
    \item \textit{matplotlib.pyplot} et \textit{seaborn} pour la création de graphiques.
    \item \textit{tqdm} pour ajouter une barre de progression de l’avancement du traitement des fichiers.
    \item \textit{math} pour utiliser des fonctions mathématiques avancées.
    \item \textit{glob} pour la rechercher des fichiers correspondant à un modèle spécifique.
    \item À compléter dès que le script du calcul de quantiles est fait. 
\end{itemize}


%%%%%%%%%%%%%%%%%%%%%%%%%%%%%%%%%%%%%%%%%%%%%%%%%%%%%%%%%%%%%%%%%%%%%%%%%%%%%%%%
%%%%%%%%%%%%%%%%%%%%%%%%%%%%%%%%%%%%%%%%%%%%%%%%%%%%%%%%%%%%%%%%%%%%%%%%%%%%%%%%

\section{Analyses}

%%%%%%%%%%%%%%%%%%%%%%%%%%%%%%%%%%%%%%%%%%%%%%%%%%%%%%%%%%%%%%%%%%%%%%%%%%%%%%%%

\subsection{Statistiques descriptives}

Bien avant de nous atteler aux analyses plus poussées comme les algorithmes de prédiction conformes ou les calculs des scores, nous avons réalisé une série de statistiques descriptives sur le premier jeu de données. 

\vspace{2.0cm}

Nous souhaitions tout d'abord voir la répartition des espèces qui sont photographiées en Europe du Sud-Ouest.

\begin{figure}[H]
  \centering
  \includegraphics[scale=0.3]{images/10_Most_Observed_Species.png}
  \label{fig1}
\end{figure}

Nous pouvons lire ci-dessus les appellations des dix espèces les plus photographiées. Ce sont essentiellement des arbres vivaces, certains fruitiers, ou des fleurs comestibles, aux propriétés curatives qui peuvent être aperçues dans toute la zone géographique qui nous intéresse. 

\vspace{0.2cm}

Le \textit{Prunus Spinosa L.} ou, plus communément, prunellier est l'espèce qui revient le plus souvent, nous la verrons apparaître dans la majorité des graphes qui vont suivre. Parmi les espèces les moins observées, nous n'avons pas retenu de particularités qui les relieraient mais il est très probable que ce soient des plantes plus rares ou qui suscitent moins l'intérêt de l'observateur.

\vspace{0.2cm}

Au-delà de la répartition des apparitions des espèces qui semble seulement impliquer un intérêt de l'observateur, nous nous sommes intéressées aux scores top $1$ apportés par l'IA de Pl@ntNet, c'est-à-dire à l'espèce prédite ayant la plus grande probabilité d'être correcte pour une photographie donnée. Nous avons donc obtenu les graphiques suivants :

\begin{figure}[H]
\centering
\begin{minipage}{0.5\textwidth}
  \includegraphics[width=0.8\linewidth]{images/mean_rd.png}
\end{minipage}%
\begin{minipage}{0.5\textwidth}
  \includegraphics[width=0.8\linewidth]{images/max_rd.png}
\end{minipage}
\end{figure}

Le premier graphe représente la moyenne des probabilités de toutes les observations pour une même espèce par rapport au nombre d'occurrences de cette dernière. Quant au second graphe, il représente le maximum de ces probabilités. Chaque point violet représente une photographie prise par un observation.
Nous avons mis en relief une espèce n'ayant été observée qu'une seule fois dans ce jeu de données, \textit{Pavonia spinifex (L.) Cav.} et le prunellier déjà cité.

\vspace{2.0cm}

À première vue, il n'y a pas de relation évidente entre la fréquence d'apparitions et les scores de l'IA. Nous remarquons tout de même que le prunellier a en moyenne un score Top 1 assez proche de 1 et que le pavonia spinifex a un score assez bas, et la répartition des points sur les deux graphes est similaire, ce sont les mêmes 2000 espèces aléatoirement désignées qui sont présentées.

\vspace{2.0cm}

Nous avons quelques hypothèses permettant de comprendre pourquoi certaines espèces vues une seule fois peuvent avoir de très bon scores et pourquoi, au contraire, certaines espèces fréquentes peuvent avoir des scores presque nuls. Tout d'abord, la qualité de l'image : si l'objectif est trop éloigné, si l'image est floue, s'il y a des objets qui font obstacle (un emballage plastique transparent par exemple) ou s'il y a plusieurs espèces dans la même image, alors la performance de l'IA en est lésée.
Ensuite, il se peut que l'IA ait été entraînée sur des images d'une espèce qui, dans notre base de données, n'apparaît qu'une seule fois. Ainsi, pour l'IA, c'est une espèce qu'elle sait reconnaître à chaque fois car elle existe dans sa base d'entraînement comme la même photographie. Ainsi, la plante peut se voir octroyer un score élevé même dans notre jeu de données.

...

%%%%%%%%%%%%%%%%%%%%%%%%%%%%%%%%%%%%%%%%%%%%%%%%%%%%%%%%%%%%%%%%%%%%%%%%%%%%%%%%

\subsection{Prédiction conforme}

La prédiction conforme permet de quantifier l'incertitude des prédictions faites par des algorithmes de prédiction arbitraire (ref). C'est-à-dire convertir les prédictions d'un algorithme en un ensemble de prédictions qui ont de fortes probabilités de contenir la réponse correcte.
% * <aigoin.emilie@gmail.com> 12:02:29 29 Mar 2025 UTC+0100:
% mettre ref

\vspace{0.2cm}

Nous avons $(X_i, Y_i)$ des paires constituées de caractéristiques ($X_i$) et de réponses ($Y_i$) indépendants et identiquement distribués issus d'une distribution $P$, avec $i = 1, \dots, n$. Notre espace des caractéristiques est $X = \mathbb R^d$ et notre espace des réponses est $Y = \mathbb R$.

\vspace{0.2cm}

Nous fixons un niveau d'erreur $\alpha \in ]0,1[$.

\vspace{0.2cm}

Ainsi, nous cherchons à construire un intervalle $\hat C_n (X_{n+1})$ qui contienne $Y_{n+1}$ avec une probabilité d'au moins $1- \alpha$, c'est-à-dire : $$ \mathbb P(Y_{n+1} \in \hat C_n (X_{n+1}) \geq 1 - \alpha) $$

\vspace{0.2cm}

Tout cela nécessite certaines conditions. La première va être de ne pas poser d'hypothèse sur $P$. Nous ne devons pas non plus utiliser toutes les prédictions possibles car nous voulons un nombre de précision fini mais également raisonnable. Pour finir, nous voulons adapter notre stratégie à la dureté du problème, c'est-à-dire que plus il est facile de prédire $Y_{n+1}$ à partir de $X_{n+1}$ et plus notre ensemble $\hat C_n(X_{n+1})$ devra être petit.

\vspace{0.2cm}

Cela est tout à fait possible avec une distribution infinie dans des conditions standard (convergence du quantile de l'échantillon vers le quantile de la population). Mais, étant donné que nous sommes dans des conditions réelles, nous nous intéressons ici à des échantillons finis.

%%%%%%%%%%%%%%%%%%%%%%%%%%%%%%%%%%%%%%%%%%%%%%%%%%%%%%%%%%%%%%%%%%%%%%%%%%%%%%%%

\subsection{Algorithmes de prédiction}

Le premier algorithme que nous avons utilisé est celui de la prédiction conforme de Vovk et al.  qui permet d'obtenir le score de la bonne classe.
% * <aigoin.emilie@gmail.com> 12:33:22 29 Mar 2025 UTC+0100:
% ajouter ref

\vspace{0.2cm}

Le deuxième algorithme dont nous nous sommes servis est celui de Romano et al. qui nous permet d'obtenir un ensemble de prédictions adaptatifs. 
% * <aigoin.emilie@gmail.com> 12:34:34 29 Mar 2025 UTC+0100:
% ajouter ref 

Pour cela, il faut tout d'abord commencer par ordonner les classes en fonction de leurs scores dans l'ordre décroissant. Puis, prendre la somme des sorties softmax jusqu'à atteindre la vraie classe (formule) (graphique).
% * <aigoin.emilie@gmail.com> 12:36:32 29 Mar 2025 UTC+0100:
% ajouter formule et graphique


%%%%%%%%%%%%%%%%%%%%%%%%%%%%%%%%%%%%%%%%%%%%%%%%%%%%%%%%%%%%%%%%%%%%%%%%%%%%%%%%
%%%%%%%%%%%%%%%%%%%%%%%%%%%%%%%%%%%%%%%%%%%%%%%%%%%%%%%%%%%%%%%%%%%%%%%%%%%%%%%%

\section{Conclusion}



\printbibliography

\end{document}

