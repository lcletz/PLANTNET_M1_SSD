\documentclass[a4paper,12pt]{article}

% Définition des packages et de leurs paramètres

\usepackage{graphicx}  % Pour insérer des images
\usepackage{textpos}   % Pour positionner précisément les images
\usepackage[french]{babel}  % Utiliser le package babel pour le français
\usepackage{csquotes}         % Ajouter csquotes pour babel
\usepackage[T1]{fontenc}      % Corriger l'encodage pour le français
\usepackage[colorlinks=true, linkcolor=blue]{hyperref}  % Liens cliquables en bleu sans les encadrer
\usepackage{amssymb}    % pour les symboles mathématiques
\usepackage{float}      % Pour le placement précis des figures avec [H]
\usepackage{microtype}

\usepackage{geometry}  % Pour ajuster les marges
\geometry{top=2cm, bottom=2cm, left=2cm, right=2cm}

% Ajout d'un espace insécable entre les décimales
\usepackage{siunitx}
\sisetup{output-decimal-marker = {,}, group-separator = {\,}, group-minimum-digits = 4}


\usepackage[style=apa, backend=biber]{biblatex}
\addbibresource{references.bib}

\usepackage{tocloft} % Pour modifier l'apparence du sommaire
\setlength{\cftbeforesecskip}{10pt} % Espacement entre les sections
\setlength{\cftbeforesubsecskip}{8pt} % Espacement entre les sous-sections
\renewcommand{\cftsecleader}{\cftdotfill{\cftdotsep}}
\setlength{\cftaftertoctitleskip}{25pt} % Espace sous le titre du sommaire

% Informations du document

\title{Projet Pl@ntnet}
\author{Émilie Aigoin}
\date{January 2025}

\begin{document}

% Logo de l'université en haut à gauche
\begin{textblock*}{3cm}(0.01cm,0.2cm)
    \includegraphics[width=4cm]{images/Universite.png}
\end{textblock*}

% Logo du master en haut à droite
\begin{textblock*}{3cm}(11cm,1cm)
    \includegraphics[width=6cm]{images/SSD.png} 
\end{textblock*}

% Titre principal
\vspace{8cm}
\begin{center}
\large{Projet de master présenté dans le contexte de l'UE HAX817X} \\ \vspace{0.4cm}
    {\LARGE \textbf{Prédiction conformelle et base de données \\ \vspace{0.4cm} Pl@ntNet-CrowdSWE}}\\[1cm]
\end{center}

% Logo de l'application centré
\begin{textblock*}{3cm}(5cm,0.5cm)
    \includegraphics[width=6cm]{images/plantenet.png}  % Remplacer par le bon fichier
\end{textblock*}

% Informations
\vfill
\begin{center}
Présenté par \\ \vspace{0.2cm}
    {\textbf{AIGOIN Emilie \\ \vspace{0.1cm} CLETZ Laura \\ \vspace{0.1cm} THOMAS Anne-Laure}}\\ \vspace{0.6cm}
    
Sous la direction (ou co-direction) de \\ \vspace{0.2cm}
    {\textbf{BOTELLA Christophe \\ \vspace{0.1cm} SALMON Joseph }}\\ \vspace{1.5cm}
    
    {\large Master Statistique et Science des Données, \\ \vspace{0.1cm} Université de Montpellier}\\ \vspace{0.6cm}
    {\large Année 2024 - 2025}
\end{center}

% Page de garde sans numéro de page
\thispagestyle{empty}

\newpage

% Insertion du sommaire
\tableofcontents 

\newpage

%%%%%%%%%%%%%%%%%%%%%%%%%%%%%%%%%%%%%%%%%%%%%%%%%%%%%%%%%%%%%%%%%%%%%%%%%%%%%%%%
%%%%%%%%%%%%%%%%%%%%%%%%%%%%%%%%%%%%%%%%%%%%%%%%%%%%%%%%%%%%%%%%%%%%%%%%%%%%%%%%

\section{Introduction}

La reconnaissance des plantes est une problématique clé en botanique, avec des applications directes dans la conservation de la biodiversité, la gestion des écosystèmes ou encore la culture personnelle. Parmi les initiatives les plus importantes dans ce domaine, le projet Pl@ntNet joue un rôle central en proposant une application de reconnaissance automatique des plantes basée sur des données collectées par les utilisateurs. Cette approche de science participative permet de constituer une base de données massive et diversifiée de plus de $6$ millions d'observations pour la seule régione de l'Europe du Sud-Ouest, impliquant plus de $\num{800000}$ contributeurs et couvrant plus de $\num{17000}$ espèces végétales. Ce qui est essentielle pour entraîner et affiner les modèles de classification.

\vspace{0.2cm}

Cependant, garantir la fiabilité des prédictions reste un défi majeur. Les données collectées, bien que nombreuses, peuvent être de qualité variable en raison des conditions de prise de vue (lumière, angle, netteté), de la diversité des espèces et de l'expertise variables des contributeurs. Pour améliorer la précision de l'application, il ne suffit pas d’optimiser la classification : nous devons également quantifier l’incertitude des prédictions et adapter dynamiquement la sortie du modèle en fonction du niveau de confiance.

\vspace{0.2cm}

Dans ce travail, nous nous sommes intégrées au projet Pl@ntNet avec un objectif précis : améliorer la fiabilité des prédictions en les rendant adaptatives. Plutôt que de fournir une unique réponse avec une probabilité associée, notre objectif était de générer un ensemble de prédictions dont la probabilité de contenir la bonne espèce atteint un niveau de garantit satisfaisant (fixé comme étant $95\%$). Cet ensemble doit être le plus restreint possible pour éviter les suggestions inutiles, tout en s'ajustant automatiquement en fonction de la difficulté de l’identification : être plus précis pour les cas évidents et plus large pour les situations ambiguës.

\vspace{0.2cm}

Pour répondre à cette problématique, nous avons utilisé la prédiction conforme, une approche statistique permettant de transformer les sorties probabilistes d'un modèle de classification en ensembles de prédiction avec des garanties de couverture. Contrairement aux méthodes classiques, la prédiction conforme offre des garanties valides même pour des échantillons de taille finie et sans hypothèses fortes sur la distribution des données.

\vspace{0.2cm}

Dans la suite de ce rapport, nous détaillerons notre approche en commençant par une présentation approfondie de l'application Pl@ntnet et des données utilisées, suivie d'analyses statistiques descriptives. Nous introduirons ensuite le cadre théorique de la prédiction conforme avant de présenter notre méthodologie, nos résultats principaux et leurs implications pour l'amélioration de l'application.

%%%%%%%%%%%%%%%%%%%%%%%%%%%%%%%%%%%%%%%%%%%%%%%%%%%%%%%%%%%%%%%%%%%%%%%%%%%%%%%%
%%%%%%%%%%%%%%%%%%%%%%%%%%%%%%%%%%%%%%%%%%%%%%%%%%%%%%%%%%%%%%%%%%%%%%%%%%%%%%%%

\section{Jeux de données et outils}

%%%%%%%%%%%%%%%%%%%%%%%%%%%%%%%%%%%%%%%%%%%%%%%%%%%%%%%%%%%%%%%%%%%%%%%%%%%%%%%%

\subsection{Application Pl@ntnet}

Pl@ntNet est un projet de sciences participatives accessible sous forme d’application mobile gratuite mais également via une \href{https://identify.plantnet.org/fr}{interface en ligne}. Développé conjointement par l'Institut National de Recherche Informatique et Automatique (INRIA), le Centre de coopération International en Recherche Agronomique (CIRAD) et l'Institut de Recherche pour le Développement (IRD), ce projet a été lancé en $2009$ et recense plus de $20$ millions d'utilisateurs dans le monde en $2024$. 

\vspace{0.2cm}

Son objectif principal est d'aider le grand public et le professionnels à identifier les espèces végétales à partir de photographies. L'application s'appuie sur des algorithmes d'intelligence artificielle qui analysent les caractéristiques visuelles des plantes (écorce, feuilles, fruits, fleurs, etc.) pour proposer des identifications. Au-delà de la reconnaissance des plantes, Pl@ntnet permet également de cartographier la distribution géographique des espèces végétales en fonction de la localisation des photographies paratgés par les utilisateurs.

\vspace{0.2cm}

Pl@ntNet est basée sur un principe d’apprentissage coopératif et itératif. Les utilisateurs peuvent partager leurs photographies (que nous appellerons des observations) et celles-ci peuvent être ensuite révisées par la communauté. Cet autre type de contribution est utilisé non seulement pour enrichir la base de données mais est aussi utilisé par l’IA pour améliorer les performances de son système de reconnaissance. Les utilisateurs peuvent, par exemple, confirmer l'identification d'une espèce, suggérer une identification alternative, ou signaler une erreur.

\vspace{0.2cm}

Le processus fonctionne ainsi : lorsqu'un utilisateur soumet une photographie, l'algorithme d'intelligence artificielle analyse l'image et généère une liste d'espèces candidates (appelées étiquettes), chacune associée à une probabilité. L'espèce affichée en première position est celle que l'intelligence artificielle estime la plus probable, suivie des autres classées par ordre décroissant de probabilité. Pour garantir la pertinence des suggestions et ne pas surcharger l'utilisateur d'informations avec des espèces peu probables, l'affichage est limité aux espèces dont la probabilité dépasse au seuil minimal, fixé à $0,001$ (soit $0,1\%$).

\vspace{0.2cm}


Notre travail est basé sur une démarche d'optimisation de ce système, avec pour objectif de rendre le nombre d'étiquettes présentées adaptatif à la difficulté du problème d'identification : plus l'identification est facile, moins le système proposera d'options à l'utilisateur, et inversement pour les cas ambigus ou difficiles.

\vspace{0.2cm}

Toutes les interactions entre les utilisateurs et le système (incluant les images soumises, les prédictions de l'intelligence artificielle et les validations humaines) sont stockées dans une base de données qui a constitué notre jeu de données tout au long de ce projet.

%%%%%%%%%%%%%%%%%%%%%%%%%%%%%%%%%%%%%%%%%%%%%%%%%%%%%%%%%%%%%%%%%%%%%%%%%%%%%%%%

\subsection{Présentation du jeu de données}

Notre étude s'est appuyée sur plusieurs jeux de données issus de Pl@ntnet, chacun présentant des caractéristiques spécifiques en termes de taille, de structure et de variables.

\vspace{0.2cm}

Dans un premier temps, nous nous sommes familiarisées avec le jeu de données principal Pl@ntnet-CrowdSWE (pour South-West Europe), comprenant $\num{6 699 593}$ observations d'espèces végétales situées en Europe du Sud-Ouest. Ces observations ont été réalisées par $\num{823 251}$ utilisateurs de l'application Pl@ntNet, parmi lesquels nous distonguons deux catégories : 
\begin{itemize}
    \item Les experts : au nombre de $98$, ce sont des botaniques professionnels ou des amateurs très expérimentés dont nous admettons la véracité des identifications.
    \item Les non-experts : constituent la majorité des contributeurs et comprennent tout les utilisateurs novices en botanique.
\end{itemize}

\vspace{0.2cm}

Cette distinction est très importante pour notre approche, car les identifications fournies par les experts servent de vérité terrain pour évaluer la qualité des prédictions automatiques et pour calibrer notre modèle de prédiction conforme.

\vspace{0.2cm}

Les données contiennent les variables suivantes :
\begin{itemize}
    \item Les identifiants uniques pour chaque utilisateur (avec leur expertise précisée).
    \item Les identifiants uniques pour chaque observation.
    \item Les identifiants pour chaque espèce (un identifiant spécifique pour la base de données mondiale et un identidiant spécifique pour la base de données SWE).
    \item Les noms scientifiques complets des espèces.
    \item Les prédictions générées par l'algorithme d'intelligence artificielle pour chaque observation avec leurs probabilités associées.
    \item Ls métadonnées sur les images (date, localisation, type d'organe photographié).
    \item Les votes des utilisateurs sur les identifications et leurs propositions sur les observations.
\end{itemize}

\vspace{0.2cm}

Le tableau ci-dessous résume les principales caractéristiques de ce jeu de données :

\vspace{0.2cm}

\begin{center}
\begin{tabular}{|c|c|}
    \hline
    Caractéristique & Valeur \\
    \hline
    Zone géographique  & Europe du Sud-Ouest  \\
    Nombre d'observations & $\num{6 699 593}$  \\
    Nombre d'utilisateurs  & $\num{823 251}$  \\
    Nombre d'utilisateurs experts  & $98$  \\
    Nombre d'espèces observées  & $\num{17 247}$  \\
    Période couverte  & $2015-2023$  \\
    \hline
    \end{tabular}
\end{center}
    
\vspace{0.2cm}

Ces données sont réparties dans $7$ fichiers au format JSON (JavaScript Object Notation) et $2$ fichiers textes disponibles sur la \href{https://zenodo.org/records/10782465}{plateforme Zenodo}, un centre de données du CERN ouvert à tous.

\vspace{0.2cm}

Dans un deuxième temps, nous avons travaillé avec un échantillon plus restreint de $\num{67 466}$ observations afin de faciliter le développement et les tests de nos algorithmes, permettant une exécution plus rapide qu'avec l'ensemble complet des $7$ millions de données. Cette approche nous a permit d'itérer efficacement sur nos modèles avant de les généraliser sur le jeu de données intégral.

%%%%%%%%%%%%%%%%%%%%%%%%%%%%%%%%%%%%%%%%%%%%%%%%%%%%%%%%%%%%%%%%%%%%%%%%%%%%%%%%

\subsection{Outils}

Pour mener à bien nos analyses, nous avons développé une chaîne de traitements combinant plusieurs langages et outils de programmation. Tous nos codes sont disponibles en libre accès sur notre \href{https://github.com/lcletz/PLANTNET_M1_SSD}{Github}.

\vspace{0.2cm}

Nous avons principalement utilisés les langages de programmation R via RStudio (\cite{RStudio}) ainsi que Python (\cite{Python}) pour le traitement des données et les analyses statistiques. Le fait d'avoir utiliser plusieurs langages nous a permis de tirer parti de leurs forces : R pour ses capacités graphiques et statistiques avancées et Python pour sa flexibilité dans la manipulation de grands volumes de données (possibilités de travailler avec des fichiers zippés).

\vspace{0.2cm}

Pour nos analyses sous R, nous avons mobilisés les packages suivants :
\begin{itemize}
    \item Manipulation des données : \texttt{jsonlite} pour la lecture et l'écriture de fichiers JSON, \texttt{tibble} et \texttt{data.table} pour optimiser le traitement de grands tableaux de données, \texttt{dplyr} et \texttt{tidyr} pour les opérations de filtrage et de croisements de données.
    \item Visualisation : \texttt{ggplot2} pour la création de graphiques, \texttt{gridExtra} pour la composition de plusieurs graphiques, \texttt{htmlwidget} et \texttt{plotly} pour générer des visualisations interactives exportables.
    \item Traitement fonctionnel : \texttt{purrr} pour appliquer des fonctions prenant en entrée des vecteurs dans les listes larges.
\end{itemize}

\vspace{0.2cm}

Pour nos analyses sous Python, les bibliothèques dont nous avons fait usage sont :
\begin{itemize}
    \item Gestion de fichiers : \texttt{requests}, \texttt{io}  et \texttt{zipfile} pour la récupération et extraction de fichiers compressés, \texttt{json} et \texttt{tarfile} pour la manipulation de fichiers JSON et TAR (archives compressées), \texttt{os} et \texttt{glob} ppour la navigation et la recherche dans l'arborescence des fichiers.
    \item Analyse de données : \texttt{pandas} pour la manipulation tabulaire des données et les opérations de fusion, \texttt{numpy} pour les calculs numériques vectorisés.
    \item Visualisation : \texttt{matplotlib.pyplot} et \texttt{seaborn} pour la création de graphiques.
    \item Optimisation : \texttt{tqdm} pour suivre visuellement la progresson des traitements, \texttt{math} pour utiliser des fonctions mathématiques avancées.
    \item À compléter dès que le script du calcul de quantiles est fait. 
\end{itemize}

\vspace{0.2cm}

Cette complémentarité des outils nous a permis d'aborder efficacement les différentes phases du projet : de l'exploration initiale des données à l'implémentation des algorithmes de prédiction conformes, en passant par la visualisation des résultats.

%%%%%%%%%%%%%%%%%%%%%%%%%%%%%%%%%%%%%%%%%%%%%%%%%%%%%%%%%%%%%%%%%%%%%%%%%%%%%%%%
%%%%%%%%%%%%%%%%%%%%%%%%%%%%%%%%%%%%%%%%%%%%%%%%%%%%%%%%%%%%%%%%%%%%%%%%%%%%%%%%

\section{Analyses}

%%%%%%%%%%%%%%%%%%%%%%%%%%%%%%%%%%%%%%%%%%%%%%%%%%%%%%%%%%%%%%%%%%%%%%%%%%%%%%%%

\subsection{Statistiques descriptives}

Bien avant de nous atteler aux analyses plus poussées comme les algorithmes de prédiction conformes ou les calculs des scores, nous avons réalisé une série de statistiques descriptives sur le premier jeu de données. Ces analyses préliminaires nous ont permis de mieux comprendre la nature des données et d'identifier certains phénomènes intéressants.

\vspace{0.2cm}

Nous avons tout d'abord examiné la répartition des espèces photographiées en Europe du Sud-Ouest.

\begin{figure}[H]
  \centering
  \includegraphics[scale=0.3]{images/10_Most_Observed_Species.png}
  \caption{Histogramme des $10$ espèces végétales les plus osbervées}
  \label{fig1}
\end{figure}

%%%%% ???? refaire les graphiques ???? %%%%%

Comme le montre la Figure $1$, les espèces les plus fréquemment photographiées par les utilisateurs de Pl@ntnet sont essentiellement des arbres vivaces, certains fruitiers, ou des fleurs comestibles, aux propriétés curatives et qui peuvent être aperçues dans toute la zone géographique étudiée. Le \textit{Prunus Spinosa L.} ou, plus communément, prunellier, arrive en tête des observations, suivi par d'autres espèces communes comme le \textit{Fagus Sylvatica L.} (hêtre commun) ou encore le \textit{Corylus avellana L.} (noisetier commun).

\vspace{0.2cm}

Cette distribution n'est pas surprenante puisqu'elle reflète à la fois l'abondance de ces espèces dans l'écosystème méditerranéen et l'intérêt qu'elle suscite auprès du grand public, soit pour leur valeur décorative, soit pour leurs usages alimentaires ou médicinaux.

\vspace{0.2cm}

A l'autre extrémité, les espèces les moins observées (avec parfois une seule occurence dans la base), correspondent généralement à des plantes rares, présentent dans des zones très restreintes, ou simplement moins reconnaissables par le grand public.

\vspace{0.2cm}

Au-delà de la simple fréquence d'apparition des espèces, qui semble seulement impliquer un intérêt de l'observateur, nous nous sommes intéressées à la relation entre cette fréquence et la qualité des prédictions de l'algorithme d'intelligence artificielle. Pour cela, nous avons analysé les scores "Top $1$" attribyés par le système, c'est-à-dire la probabilité associée à l'espèce la plus probable selon l'algorithme. Nous avons obtenu les graphiques suivants :

\begin{figure}[H]
\centering
\begin{minipage}{0.5\textwidth}
  \includegraphics[width=0.8\linewidth]{images/mean_rd.png}
\end{minipage}%
\begin{minipage}{0.5\textwidth}
  \includegraphics[width=0.8\linewidth]{images/max_rd.png}
\end{minipage}
\caption{Nuages de points de la moyenne des scores ou des scores max en fonction du nombre d'observations}
\end{figure}

%%%%% ???? refaire les graphiques ???? %%%%%

Ces deux graphiques révèlent plusieurs phénomènes interéssants.

Contrairement à ce que l'on pourrait intuitivement penser, il n'existe pas de relation linéaire évidente entre la fréquence d'observation d'une espèce et la confiance moyenne du système dans ses prédictions. Des espèces très fréquentes peuvent recevoir des scores moyens relativement bas, tandis que certaines espèces rares obtiennent des scores élevées.

\vspace{0.2cm}

Le \textit{Prunus Spinosa L.}
, l'espèce la plus observées, présente un score moyen relativement élevé (proche de $0,6$), ce qui suggère que le système est généralement confiant dans son identification. Cela peut s'expliquer par ses caractéristiques morphologiques distinctives et possiblement par la qualité généralement bonne des photos de cette espèce commune.

\vspace{0.2cm}

Certaines espèces n'ayant qu'une seule observation présentent des scores très élevés, proches de $1$. Ce phénomène pourrait s'expliquer par le processus d'apprentissage du modèle : si l'intelligence artificielle a été entraînée sur des images similaires à cette unique observation dans notre jeu de données, elle peut la reconnaître avec une grande confiance.

\vspace{0.2cm}

Plusieurs facteurs peuvent expliquer la variabilité observée dans les scores de prédiction : 
\begin{itemize}
    \item Qualité des images : les photos floues, mal cadrées, prises de trop loin ou dans des conditions d'éclairage défavorables résuident considérablement la performance de l'algorithme.
    \item Confusion entre espèces similaires : certaines espèces appartenant au même genre partagent des caractéristiques morphologiques très proches, ce qui peut induire l'algorithme en erreur. Par exemple, différentes espèces de chênes ou de roses peuvent être difficiles à distinguer même pour des botanistes expérimentés.
    \item Stade de développement de la plante : une même espèce peut présenter des aspects très différents selon la saison (avec ou sans feuilles, en fleur ou non, avec ou sans fruit, etc.), ce qui peut affecter la confiance du système dans ses prédictions.
    \item Organe photographié : les fleurs sont généralement plus distinctives et permettent une identification plus fiables que les feuilles ou les tiges, qui peuvent présenter plus de similarités entre espèces.
    \item Effet d'apprentissage sur des images spécifiques : comme mentionné précédemment, certaines espèces rares peuvent obtenir des scores très élevés si l'image soumise est similaire à celle utilisée pour l'entraînement du modèle.
\end{itemize}

\vspace{0.2cm}

Ces observations préliminaires nous ont confirmé la nécessaité d'une approche adaptative pour la présentation des résultats aux utilisateurs. En effet, un système qui proposerait un nombre fixe d'expèces candidates serait soit top restrictif dans les cas difficiles (risquant d'oublier la bonne espcèce), soit trop verbeux dans les cas simples (noyant l'utilisateur sous des propositions inutiles).

\vspace{0.2cm}

La prédiction conforme, que nous allons introduire dans la section suivante, offre plus précisément le cadre mathématique nécessaire pour adapter dynamiquement le nombre de suggestions en fonction de la difficulté de chaque cas d'identification.

%%%%%%%%%%%%%%%%%%%%%%%%%%%%%%%%%%%%%%%%%%%%%%%%%%%%%%%%%%%%%%%%%%%%%%%%%%%%%%%%

\subsection{Prédiction conforme}

La prédiction conforme permet de quantifier l'incertitude des prédictions faites par des algorithmes de prédiction arbitraire (ref). C'est-à-dire convertir les prédictions d'un algorithme en un ensemble de prédictions qui ont de fortes probabilités de contenir la réponse correcte.
% * <aigoin.emilie@gmail.com> 12:02:29 29 Mar 2025 UTC+0100:
% mettre ref

\vspace{0.2cm}

Nous avons $(X_i, Y_i)$ des paires constituées de caractéristiques ($X_i$) et de réponses ($Y_i$) indépendants et identiquement distribués issus d'une distribution $P$, avec $i = 1, \dots, n$. Notre espace des caractéristiques est $X = \mathbb R^d$ et notre espace des réponses est $Y = \mathbb R$.

\vspace{0.2cm}

Nous fixons un niveau d'erreur $\alpha \in ]0,1[$.

\vspace{0.2cm}

Ainsi, nous cherchons à construire un intervalle $\hat C_n (X_{n+1})$ qui contienne $Y_{n+1}$ avec une probabilité d'au moins $1- \alpha$, c'est-à-dire : $$ \mathbb P(Y_{n+1} \in \hat C_n (X_{n+1}) \geq 1 - \alpha) $$

\vspace{0.2cm}

Tout cela nécessite certaines conditions. La première va être de ne pas poser d'hypothèse sur $P$. Nous ne devons pas non plus utiliser toutes les prédictions possibles car nous voulons un nombre de précision fini mais également raisonnable. Pour finir, nous voulons adapter notre stratégie à la dureté du problème, c'est-à-dire que plus il est facile de prédire $Y_{n+1}$ à partir de $X_{n+1}$ et plus notre ensemble $\hat C_n(X_{n+1})$ devra être petit.

\vspace{0.2cm}

Cela est tout à fait possible avec une distribution infinie dans des conditions standard (convergence du quantile de l'échantillon vers le quantile de la population). Mais, étant donné que nous sommes dans des conditions réelles, nous nous intéressons ici à des échantillons finis.

%%%%%%%%%%%%%%%%%%%%%%%%%%%%%%%%%%%%%%%%%%%%%%%%%%%%%%%%%%%%%%%%%%%%%%%%%%%%%%%%

\subsection{Algorithmes de prédiction}

Le premier algorithme que nous avons utilisé est celui de la prédiction conforme de Vovk et al.  qui permet d'obtenir le score de la bonne classe.
% * <aigoin.emilie@gmail.com> 12:33:22 29 Mar 2025 UTC+0100:
% ajouter ref

\vspace{0.2cm}

Le deuxième algorithme dont nous nous sommes servis est celui de Romano et al. qui nous permet d'obtenir un ensemble de prédictions adaptatifs. 
% * <aigoin.emilie@gmail.com> 12:34:34 29 Mar 2025 UTC+0100:
% ajouter ref 

Pour cela, il faut tout d'abord commencer par ordonner les classes en fonction de leurs scores dans l'ordre décroissant. Puis, prendre la somme des sorties softmax jusqu'à atteindre la vraie classe (formule) (graphique).
% * <aigoin.emilie@gmail.com> 12:36:32 29 Mar 2025 UTC+0100:
% ajouter formule et graphique


%%%%%%%%%%%%%%%%%%%%%%%%%%%%%%%%%%%%%%%%%%%%%%%%%%%%%%%%%%%%%%%%%%%%%%%%%%%%%%%%
%%%%%%%%%%%%%%%%%%%%%%%%%%%%%%%%%%%%%%%%%%%%%%%%%%%%%%%%%%%%%%%%%%%%%%%%%%%%%%%%

\section{Conclusion}



\printbibliography

\end{document}

